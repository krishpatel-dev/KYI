\chapter{System Design}

\section{System Architecture}
The Know Your Ingredients app follows a client-server architecture with a combination of cloud-based and local storage solutions. The system consists of the following major components:
\begin{enumerate}
    \item \textbf{Frontend (Android Client – Kotlin)}
    \begin{itemize}
        \item Handles user interactions such as image capture, ingredient search, and result display.
        \item Integrates Google ML Kit OCR for text extraction.
        \item Displays ingredient information retrieved from the database.
    \end{itemize}
    \item \textbf{Backend Services}
    \begin{itemize}
        \item Firebase Firestore (Cloud-based database for ingredient details, updated in real-time).
        \item CSV file (Local database) for offline access to ingredient data.
    \end{itemize}
    \item \textbf{OCR \& Data Processing}
    \begin{itemize}
        \item Uses Google ML Kit OCR for text extraction from images.
        \item Preprocessing module cleans and standardizes ingredient names.
        \item Search algorithm matches extracted text with the ingredient database.
    \end{itemize}
    \item \textbf{Data Storage \& Management}
    \begin{itemize}
        \item Ingredient data is stored and updated in Firebase Firestore.
        \item Users can access previously searched ingredients using CSV database (offline mode).
    \end{itemize}
\end{enumerate}

\section{High-Level System Design}
The high-level architecture can be visualized as follows:
\begin{figure}[h!]
    \centering
    \includegraphics[width=0.4\textwidth]{architecture_diagram.png}
    \caption{Architecture Diagram}
    \label{fig:architecture-diagram}
\end{figure}

\section{Module Breakdown}
The system is divided into five core modules, each responsible for different functionalities:
\subsection{Image Scanning \& OCR Module}
\begin{itemize}
    \item Captures images from the camera or gallery.
    \item Uses Google ML Kit OCR to extract ingredient text.
    \item Sends extracted text to the preprocessing module.
\end{itemize}

\subsection{Text Preprocessing \& Matching Module}
\begin{itemize}
    \item Cleans and standardizes extracted text (removes extra spaces, converts to lowercase).
    \item Matches ingredient names against the database using a search algorithm.
\end{itemize}

\subsection{Database Management Module}
\begin{itemize}
    \item Stores and retrieves ingredient data from Firebase (online) and SQLite (offline).
    \item Regularly syncs with Firebase to update the latest ingredient safety information.
\end{itemize}

\subsection{UI/UX Module}
\begin{itemize}
    \item Provides an intuitive interface for scanning, viewing results, and searching ingredients.
    \item Displays ingredient safety categories (Safe, Cautionary, Harmful).
    \item Includes dark mode support for accessibility.
\end{itemize}

\section{Technology Stack}
The technologies used in the system are summarized in the table below:
\begin{table}[h]
    \centering
    \begin{tabular}{|l|l|}
        \hline
        \textbf{Component} & \textbf{Technology Used} \\ \hline
        Frontend (UI/UX) & Kotlin, Jetpack Compose / XML \\ \hline
        OCR Engine & Google ML Kit OCR \\ \hline
        Database (Online) & Firebase Firestore \\ \hline
        Database (Offline) & CSV file \\ \hline
        Cloud Storage & Firebase Storage (for images) \\ \hline
        Barcode Scanning (Future) & Google Barcode API \\ \hline
        Backend Integration & REST API (for future external sources) \\ \hline
    \end{tabular}
    \caption{Technology Stack for Know Your Ingredients App}
    \label{tab:tech_stack}
\end{table}

\section{Data Flow Diagram (DFD - Level 1)}
The data flow process is as follows:
\begin{figure}[h!]
    \centering
    \includegraphics[width=0.5\textwidth]{data_flow_diagram.png}
    \caption{Data Flow Diagram}
    \label{fig:data-flow-diagram}
\end{figure}

\section{Database Design}
\subsection{Conceptual Database Design}
The database is structured as a relational model (for CSV file) and a document-based NoSQL model (for Firebase Firestore).

\subsubsection{Key Entities \& Relationships}
\begin{itemize}
    \item \textbf{Ingredients Entity}: Stores the master ingredient list with attributes such as name, benefits, risks, safety classification, and alternatives.
    \item \textbf{Products Entity}: Stores scanned product details and links them to ingredient lists.
    \item \textbf{Product-Ingredients Relationship}: Maps products to their ingredients (many-to-many relationship).
    \item \textbf{User Settings Entity (Optional)}: Stores user preferences (e.g., dietary restrictions, offline mode).
\end{itemize}

\subsection{Safety Classification System}
Ingredients are categorized into three levels based on research:
\begin{table}[h]
    \centering
    \begin{tabular}{|l|l|}
        \hline
        \textbf{Safety Category} & \textbf{Description} \\ \hline
        Safe & No known harmful effects. \\ \hline
        Cautionary & Limited research or minor potential risks. \\ \hline
        Harmful & Linked to toxicity, allergies, or health risks. \\ \hline
    \end{tabular}
    \caption{Safety Classifications for Ingredients}
    \label{tab:safety_classifications}
\end{table}

\section{Security Considerations}
\begin{table}[H]
    \centering
    \begin{tabular}{|p{3cm}|p{8cm}|}
        \hline
        \textbf{Feature} & \textbf{Description} \\ \hline
        User Privacy & No personal data is collected; only scanned ingredient text is processed. \\ \hline
        Data Protection & Firebase uses encrypted connections (SSL/TLS) for secure data transfer. \\ \hline
        Permissions Management & The app will request camera and storage permissions only when necessary. \\ \hline
    \end{tabular}
    \caption{Security and Privacy Features}
    \label{tab:security_privacy}
\end{table}

\section{User Interface Design}
\subsection{Main Screens}
\begin{table}[h]
    \centering
    \begin{tabular}{|l|l|}
        \hline
        \textbf{Screen} & \textbf{Description} \\ \hline
        Home Screen & Camera \& gallery upload options. \\ \hline
        Scanning Screen & OCR processing in progress. \\ \hline
        Results Screen & Ingredient list with safety classifications. \\ \hline
        Ingredient Details Screen & In-depth analysis of individual ingredients. \\ \hline
        Settings Screen & Language preferences, offline mode toggle. \\ \hline
    \end{tabular}
    \caption{Screens of the Know Your Ingredients App}
    \label{tab:app_screens}
\end{table}

\begin{figure}[h!] 
    \centering
    \vspace{1cm}
    \includegraphics[width=\textwidth]{Highlevelsystemarch.png}
    \caption{High-Level System Architecture}
    \label{fig:high-level-architecture}
\end{figure}
