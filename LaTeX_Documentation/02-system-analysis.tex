\chapter{System Analysis}

\section{Literature Survey}
\subsection{Overview of Ingredient Analysis Applications}
Over the years, several mobile applications have been developed to help users analyze product ingredients. Some of the most popular ones include:
\begin{itemize}
    \item \textbf{Yuka} – Focuses on both food and cosmetics but lacks a comprehensive database.
    \item \textbf{Think Dirty} – Specializes in beauty and cosmetic product analysis but does not include food items.
    \item \textbf{Fooducate} – Focuses on analyzing food products and providing health scores but lacks beauty product analysis.
\end{itemize}
While these applications are useful, they have certain limitations, such as:
\begin{enumerate}
    \item Limited ingredient databases
    \item Poor text recognition
    \item Lack of combined analysis for food and beauty products
\end{enumerate}
Our project aims to address these gaps by providing a more comprehensive and accurate solution.

\subsection{OCR and Machine Learning in Mobile Applications}
OCR (Optical Character Recognition) is a critical technology that allows the application to scan and recognize text from images. Common OCR tools include:
\begin{itemize}
    \item \textbf{Google ML Kit OCR} – A cloud-based and on-device OCR tool with high accuracy.
    \item \textbf{Tesseract OCR} – An open-source OCR engine commonly used in mobile applications.
\end{itemize}
By combining OCR with Machine Learning, we can enhance text recognition and improve ingredient classification accuracy.

\subsection{Health Implications of Harmful Ingredients}
Several food additives and cosmetic chemicals are known to have health risks:
\begin{itemize}
    \item \textbf{Parabens} – Found in cosmetics; linked to hormonal imbalances.
    \item \textbf{Artificial Sweeteners} – Found in diet foods; associated with metabolic disorders.
    \item \textbf{Synthetic Fragrances} – Found in beauty products; can cause allergic reactions.
\end{itemize}
Providing accurate and scientific information on such ingredients is essential for public awareness.

% Including the flowchart image
\begin{figure}[h!]
    \centering
    \includegraphics[width=\textwidth]{flowchart_process.png}
    \caption{Working process of the application}
    \label{fig:literature-survey-flowchart}
\end{figure}

\section{Software Requirements Specification (SRS)}
\subsection{Overall Description}
\subsubsection{Product Perspective}
The application serves as a standalone consumer tool for ingredient analysis, utilizing OCR technology to extract ingredient names from product labels and cross-referencing them with a structured database. It integrates with Google ML Kit for OCR and Firebase/SQLite for ingredient data storage and retrieval.

\subsubsection{Product Features}
\begin{enumerate}
    \item \textbf{OCR-Based Ingredient Scanning} – Extracts text from images of ingredient lists.
    \item \textbf{Ingredient Database} – Provides descriptions, benefits, risks, and safety classifications.
    \item \textbf{Categorization System} – Labels ingredients as Safe, Cautionary, or Harmful.
    \item \textbf{Offline Functionality} – Uses CSV database for local ingredient storage.
    \item \textbf{User-Friendly UI} – Displays ingredient details and product safety summaries.
\end{enumerate}

\subsubsection{User Classes and Characteristics}
\begin{table}[h]
    \centering
    \begin{tabular}{|l|l|}
        \hline
        \textbf{User Category} & \textbf{Description} \\ \hline
        General Consumers & Individuals seeking ingredient safety insights. \\ \hline
        Health-Conscious Users & Users with allergies, dietary restrictions, or sensitivities. \\ \hline
        Researchers \& Professionals & Users analyzing product formulations. \\ \hline
    \end{tabular}
    \caption{Target User Groups for Know Your Ingredients App}
    \label{tab:user_groups}
\end{table}

\subsubsection{Operating Environment}
\begin{itemize}
    \item Android devices running Android 7.0 (Nougat) and above.
    \item Requires internet connectivity for Firebase database access.
    \item Functions offline using CSV database for ingredient lookup.
\end{itemize}

\subsection{Functional Requirements}
\subsubsection{Image Scanning \& OCR}
\begin{itemize}
    \item The app must allow users to capture images from the camera or gallery.
    \item Extracted text should be processed and formatted for accuracy.
\end{itemize}

\subsubsection{Ingredient Lookup \& Categorization}
\begin{itemize}
    \item The app must search for extracted ingredients in the database.
    \item Ingredients should be categorized as Safe, Cautionary, or Harmful.
    \item Users should be able to view ingredient details, including pros and cons.
\end{itemize}

\subsubsection{Database Management}
\begin{itemize}
    \item Firebase should handle cloud-based ingredient storage and updates.
    \item CSV file should provide offline access for stored ingredient data.
\end{itemize}

\subsection{Non-Functional Requirements}
\subsubsection{Performance Requirements}
\begin{itemize}
    \item The app should extract text and display results within 5 seconds.
    \item Offline functionality must support at least 500 stored ingredients.
\end{itemize}

\subsubsection{Security Requirements}
\begin{itemize}
    \item User data should be stored securely using Firebase Authentication.
    \item Ingredient database updates should be verified before deployment.
\end{itemize}

\subsubsection{Usability Requirements}
\begin{itemize}
    \item The UI should be intuitive, accessible, and visually appealing.
    \item Multi-language support will be added in future updates.
\end{itemize}
