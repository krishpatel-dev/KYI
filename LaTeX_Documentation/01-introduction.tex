\chapter{INTRODUCTION}\doublespacing

\section{Project Background}
In recent years, consumers have become increasingly conscious about the food they eat and the beauty products they use. While many brands emphasize natural and safe ingredients, a significant number of products contain additives, preservatives, and chemicals that may have adverse health effects. However, understanding ingredient labels can be challenging due to scientific terminologies, chemical names, and lack of contextual information.

For example, ingredients such as Sodium Benzoate, BHA (Butylated Hydroxyanisole), and Parabens are common in packaged foods and beauty products, but most consumers are unaware of their effects. While some of these ingredients are safe within certain limits, excessive or prolonged exposure can pose health risks such as hormonal imbalances, allergic reactions, or even chronic diseases.

The Know Your Ingredients project aims to address this issue by developing an Android-based application using Kotlin that allows users to scan an ingredient list from any food or beauty product and receive instant feedback on the pros and cons of each ingredient. The application will classify ingredients as safe, cautionary, or harmful, based on scientific research, regulatory guidelines, and existing databases. This tool will help users make informed decisions about the products they consume and use.

\section{Problem Statement}
Consumers face several challenges when it comes to understanding product ingredients:
\begin{enumerate}
    \item \textbf{Complex Ingredient Names:} Many chemical compounds and additives are listed using scientific names, which are difficult for a layperson to understand.
    \item \textbf{Lack of Reliable Information:} Most consumers do not have access to well-researched, easy-to-understand information about the ingredients in their daily products.
    \item \textbf{Manual Research is Inefficient:} Searching for each ingredient online can be time-consuming and may lead to unreliable sources.
    \item \textbf{Health Concerns:} Some ingredients have known health risks, but consumers are often unaware of their potential effects.
    \item \textbf{Existing Solutions Have Limitations:} Current apps either focus on only food or only beauty products, lack a comprehensive database, or do not provide accurate ingredient analysis.
\end{enumerate}
To solve these challenges, our project proposes a smart and efficient mobile application that automatically extracts ingredient names, searches a structured database, and presents users with a detailed yet easy-to-understand analysis.

\section{Scope of the Project}
The Know Your Ingredients app will be designed with the following features and functionalities:
\begin{itemize}
    \item \textbf{Support for Food \& Beauty Products:} Unlike most existing apps, this project will analyze both food and beauty product ingredients.
    \item \textbf{OCR-Based Ingredient Scanning:} The app will use Google ML Kit OCR to extract text from images of ingredient lists.
    \item \textbf{Real-Time Ingredient Analysis:} The application will fetch ingredient information from a structured database and classify each ingredient into safe, cautionary, or harmful categories.
    \item \textbf{Comprehensive Database:} The app will maintain an up-to-date database containing thousands of ingredients with detailed explanations.
    \item \textbf{User-Friendly Interface:} Results will be displayed in a visually appealing, easy-to-read format with icons, color-coded indicators, and concise descriptions.
    \item \textbf{Offline Support:} Users will be able to access basic ingredient analysis even without an internet connection.
\end{itemize}

\section{Objectives}
The key objectives of the project are:
\begin{enumerate}
    \item To develop an Android application using Kotlin that allows users to scan and analyze ingredients from product packaging.
    \item To implement Optical Character Recognition (OCR) to extract text from images of ingredient lists.
    \item To build a structured and reliable ingredient database containing relevant information about food and cosmetic ingredients.
    \item To categorize ingredients based on their potential health impact (safe, cautionary, harmful).
    \item To design a user-friendly interface that provides a seamless experience for users.
    \item To ensure high accuracy in ingredient recognition and classification.
    \item To make the app lightweight and efficient, ensuring smooth performance on different Android devices.
\end{enumerate}

\section{Expected Outcomes}
By the end of the project, the following outcomes are expected:
\begin{itemize}
    \item A fully functional Android application capable of scanning and analyzing product ingredient lists.
    \item A large and structured database containing extensive information on food ingredients.
    \item A fast and reliable classification system that accurately determines ingredient safety levels.
    \item A user-friendly interface that simplifies ingredient analysis.
    \item The ability for consumers to make more informed decisions about the products they consume and use.
\end{itemize}
