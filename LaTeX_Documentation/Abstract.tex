\thispagestyle{empty}
\newpage
\cleardoublepage\phantomsection
\addcontentsline{toc}{chapter}{Abstract}\index{Abstract}
\begin{center}
{\Large \bf Abstract}\\
\end{center}
\vspace{10pt}

In today’s health-conscious world, consumers are becoming increasingly aware of the importance of understanding the ingredients in the products they consume and use daily. However, ingredient lists on food, cosmetics, and personal care products often contain complex chemical names that are difficult for the average consumer to interpret. This lack of transparency can lead to uncertainty regarding the safety, benefits, and potential risks associated with various ingredients. To address this challenge, Know Your Ingredients is an Android application developed in Kotlin that enables users to easily analyze product ingredients through a simple scanning process. By leveraging Optical Character Recognition (OCR) technology, specifically Google’s ML Kit, the app extracts text from images of ingredient lists on product packaging. The extracted text is then cross-referenced with a comprehensive ingredient database, which provides users with detailed information about each ingredient, including its common uses, health benefits, possible side effects, and safety classifications based on scientific research and regulatory guidelines.

The application classifies ingredients into three primary categories to help users quickly assess their safety: Safe, which includes ingredients that are widely accepted as non-harmful and suitable for regular use; Cautionary, which consists of ingredients that may have potential risks and should be used in moderation or under specific conditions; and Harmful, which covers substances associated with significant health concerns, including allergens, carcinogens, or ingredients restricted by regulatory bodies. The methodology behind the application consists of multiple stages. First, a comprehensive dataset is collected from authoritative sources such as scientific literature, regulatory organizations (e.g., FDA, EWG, WHO), dermatology and nutrition experts, and consumer safety advocacy groups. The second stage involves OCR-based text recognition and data extraction, where the application scans and accurately captures ingredient names from product packaging. This is followed by data preprocessing and standardization, where the extracted text is cleaned and matched against the ingredient database to ensure accuracy. Next, the ingredient classification process takes place, in which each ingredient is categorized based on toxicity levels, allergenic potential, and regulatory restrictions. The final stage focuses on UI/UX development, ensuring that the application presents ingredient information in a clear, concise, and user-friendly manner. The app provides a summary of the product’s overall safety profile, highlighting key concerns and offering insights into ingredient safety.

To efficiently manage ingredient data, Know Your Ingredients uses an offline dataset stored in CSV format and integrated with Room Database, ensuring fast and reliable access without requiring an internet connection. Firebase services are used in parallel for other features: Authentication for secure login and signup, Firestore for storing user profile details, and Firebase Storage for managing profile photos. The final output is a comprehensive product safety report, displayed through an intuitive interface that provides ingredient descriptions, benefits and risks, and an overall assessment of the product’s safety level.

To enhance the user experience further, the application will introduce several advanced features in future updates. These include community-driven ingredient reviews, where users can share their experiences and rate ingredient safety. 

By bridging the gap between technical ingredient data and everyday consumers, Know Your Ingredients aims to empower users with instant access to reliable ingredient analysis, eliminating the need for time-consuming manual searches. The application aspires to become an essential tool for health-conscious consumers, enabling them to make informed purchasing decisions by understanding the potential effects of the products they use. As concerns over product safety and ingredient transparency continue to grow, Know Your Ingredients provides a practical, science-backed solution that fosters awareness, safety, and smarter consumer choices.\\

Keywords: Android Application, Kotlin, Optical Character Recognition (OCR), Google ML Kit, Ingredient Database, Data Collection, Regulatory Guidelines (FDA, EWG, WHO), Text Recognition, Data Preprocessing, UI/UX Development, Room Database, Firebase Authentication, Firestore, Firebase Storage
